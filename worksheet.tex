\documentclass{dcbl/challenge}


%%% -- {
%%% --     "metastyle": "0.0.1",
%%% --     "meta": {
%%% --          "uuid": "92e5a3bc-4da7-4fde-89f8-5c8fe911c1e1",
%%% --          "title": "First Steps with VHDL",
%%% --          "language": "en",
%%% --          "authors": [
%%% --              {
%%% --                  "name": "Stephan Bökelmann",
%%% --                  "email": "sboekelmann@ep1.rub.de",
%%% --                  "affiliation": "AG Physik der Hadronen und Kerne"
%%% --              },
%%% --              {
%%% --                  "name": "Tabea Röthemeyer",
%%% --                  "email": "tabea.roethemeyer@gruppe.ai",
%%% --                  "affiliation": "Auto-Intern GmbH"
%%% --              }
%%% --          ],
%%% --          "depends on": {
%%% --              "worksheet": ["92e5a3bc-4da7-4fde-89f8-5c8fe911c1e1"]
%%% --          },
%%% --          "recommends": [
%%% --              {
%%% --                  "type": "website",
%%% --                  "title": "NANDland",
%%% --                  "url": "www.nandland.com",
%%% --                  "note": "A website explaining VHDL"
%%% --              },
%%% --              {
%%% --                  "type": "book",
%%% --                  "title": "VHDL-Synthese",
%%% --                  "author": "Reichardt, J. and Schwarz, B.",
%%% --                  "publisher": "de Gruyter",
%%% --                  "address": "Oldenburg",
%%% --                  "note": "A great book on VHDL",
%%% --                  "url": "https://www.amazon.de/VHDL-Synthese-Entwurf-digitaler-Schaltungen-Systeme/dp/3486581929",
%%% --                  "ISBN": "978-3486581928"
%%% --              }
%%% --          ],
%%% --          },
%%% --          "tags": {
%%% --              "primary": ["VHSIC", "IEEE Std 1076-1987"],
%%% --              "secondary": ["verilog"],
%%% --              "category": ["digital-electronics"],
%%% --         },
%%% --     }
%%% -- }

\setdoctitle{First Steps with VHDL}
\setdocauthor{Stephan Bökelmann}
\setdocemail{sboekelmann@ep1.rub.de}
\setdocinstitute{AG Physik der Hadronen und Kerne}
\usepackage{listings}

\begin{document}

As the complexity of digital electronics and Boolean logic increases, it becomes progressively more challenging to express ideas precisely. 
This is particularly true in the realm of sequential logic, where Karnaugh maps (K-Maps) reach their limits. 
To address this challenge, the VHSIC program initiated by the US Department of Defense set out to develop a precise grammar for describing digital circuits. 
VHDL, standing for \textit{Very High Speed Integrated Circuit Hardware Description Language}, emerged as a result. 
Despite being termed a \textit{language}, VHDL is not a programming language in the conventional sense but rather a tool for the detailed depiction of real circuits.
Since its standardization by the IEEE in 1987, it has played a pivotal role in digital system development.

The use of hardware description languages like VHDL enables the achievement of two primary objectives:

\begin{enumerate}
    \item Simulations provide a tangible sense of how a circuit operates, which is invaluable, especially during the design phase.
    \item Based on the VHDL description, a detailed design can be created, which is then transformable into a netlist for an actual circuit.
\end{enumerate}

This worksheet introduces the process of progressing from the idea of a digital circuit to a simulation, the results of which can be depicted in diagram form.



\section*{Exercises}
\begin{aufgabe}
    In order for us, to see the output of our simulations as a diagram in the terminal, we need to install \textbf{VCD}.
    Clone the repository of \textbf{VCD} which is linked in the annotations.
    Use the provided \texttt{README.md}, to compile the project and install it.
    Furthermore, we need an analyzer and simulator, to transform our textfile into value-change-dump (\texttt{.vcd}) file, that can be displayed.
    Thus we need to install \textbf{GHDL} by running \texttt{sudo apt install ghdl}.
\end{aufgabe}
\begin{aufgabe}
    Make a fresh directory and navigate into it. 
    Open a new file, and call it \texttt{process.vhd}.
    As a first step, we will not start with writing a new hardware-description, but with the syntax of describing a simulation.
    Usually, a simulation consists of a sequence of input-values, that are being given to a circuit in a specific order with a specific timing.
    So we want to define a sequence of values, that are being provided and displayed, but without any additional circuit connected.
    Take a look at the following code:
    \begin{lstlisting}[language=VHDL]
    library IEEE;
    use IEEE.STD_LOGIC_1164.ALL;

    entity testbench1 is
            -- definition of inputs
            -- definition of outputs
    end testbench1;

    architecture sim of testbench is
            signal a : STD_LOGIC := '0';
            signal b : STD_LOGIC := '0';
    begin
            sim_process: process                                                                                                                          
            begin
                wait for 5 s;
                a <= not a;
                wait for 5 s;
                b <= not b;
                wait for 5 s;
                a <= not a;
            end process;
    end sim;
    \end{lstlisting}
    You may also clone this code by running:\\ 
    \hspace{5cm} \texttt{git clone} \url{git@gist.github.com:772fd1357600e207e668889fdfbc76c1.git}
    \\The code consists of three sections:
    \begin{enumerate}
        \item The library declaration
        \item The entity declaration
        \item The architecture declaration
    \end{enumerate}
    The library declaration is similar to most programming languages.
    In this case we need the \texttt{STD\_LOGIC\_1164} library, to use the \texttt{STD\_LOGIC} type, which represents a boolean value.
    The next two sections define a black box, with its in- and outputs and a name, for this given block.
    Since we are looking at a \textit{testbench} which doesn't get any input from the outside and doesn't send any output, the definition is basically empty. 
    Comments in VHDL start with the a double \texttt{-} symbol.
    A template for a block that can be filled with logic is also called an entity and has to have a unique name. 
    The name \texttt{testbench} is a common name for such an entity without in and outputs, but it could be any string, that starts with a letter.
    The other section defines what is happening inside of our defined block.
    It starts with the keyword \texttt{architecture} and a name. 
    The name \texttt{sim} is a common name for such an architecture without in and outputs, but it could be any string, that starts with a letter.
    It needs to reference, which block it should implement, therefore it says in our code \texttt{of testbench}. 
    Inside we start with the definition of local variables used in this block in our case, we just want to use the \texttt{a} and \texttt{b} signals.
    The \texttt{begin} and \texttt{end} keywords are used to define the boundary of our actual implementation.
    In an architecture, there can be multiple circuits with different names, as well as subcircuits.
    We want to define just one circuit with the name \texttt{sim\_process} and of the type \texttt{process}.
    \texttt{process} is a special type of block, that can only be used in a simulation. 
    We use the operators \texttt{begin} and \texttt{end} to define the boundary of our implementation.
    The code inside of our process should be trivial to read.
    Save the code as \texttt{process.vhd}.
    The compilation and simulation are now done in three separate steps:
    \begin{enumerate}
        \item Analyze - making sure the syntax is correct
        \item Elaborate - compiling the code
        \item Simulate - calculate the result
    \end{enumerate}
    At the end of this, we receive a \texttt{.vcd} file, which can be opened in the VCD Viewer.
    In order to do these three steps, run the following commands:
    \begin{lstlisting}
    1. ghdl -a process.vhd
    2. ghdl -e -fsynopsys testbench1
    3. ghdl -r testbench1 --vcd=output.vcd --stop-time=25s
    \end{lstlisting}
    After the simulation we can take a look at the output in the \texttt{.vcd} file by running \texttt{vcd < output.vcd}.
\end{aufgabe}

\begin{aufgabe}
    We now want to add an actual logic-block to our simulation and see how our defined logic works.
    Start a second file and call it \texttt{and\_gate.vhd}.
    Fill it with the following code:
    \begin{lstlisting}[language=VHDL]
    library IEEE;
    use IEEE.STD_LOGIC_1164.ALL;
    
    entity and_gate is
        Port (  in1 : in  STD_LOGIC;
                in2 : in  STD_LOGIC;
                Q : out STD_LOGIC);
    end and_gate;
    
    architecture Behavioral of and_gate is
    begin
        Q <= in1 and in2;
    end Behavioral;
    \end{lstlisting}
    We see the same pattern as in the testbench, only that our logic now has inputs and outputs, both of the data-type \texttt{STD\_LOGIC}.
    Save the code as \texttt{and\_gate.vhd}.
    Since this is now defined as an entity, basically the same thing as if you'd have an IC laying around, we need to add it to our actual circuit.
    Open the \texttt{process.vhd} again and modify the \texttt{sim\_process} block in the following way:
    \begin{lstlisting}
    library IEEE;
    use IEEE.STD_LOGIC_1164.ALL;

    entity testbench is
    end testbench;

    architecture sim of testbench is
            signal a : STD_LOGIC := '0';
            signal b : STD_LOGIC := '0';
            signal q : STD_LOGIC;
    begin
        uut: entity work.and_gate
        port map(   a => in1,
                    b => in2,
                    q => Q);

            sim_process: process
            begin
                    wait for 5 s;
                    a <= not a;
                    wait for 5 s;
                    b <= not b;
                    wait for 5 s;
                    a <= not a;
            end process;
    end sim;
    \end{lstlisting}
    You may also just clone the following repository: \url{git@gist.github.com:3b49170db45ffaf8aa3d6c38fbae5fbe.git}.
    Can you spot the changes? 
    We added an actual instantiation of our \texttt{and\_gate} here, as well as an output to our signals. 
    After the analyzing-step, our analyzed entity \texttt{and\_gate} will be put into a temporary library, thus we have to call it by the name \texttt{work.and\_gate}.
    With the instantiation of our block, we have to connect the terminals of that entity to our signals inside our process.
    We do that by saying \texttt{a => in1}, which means, our signal \texttt{a} is connected to the terminal \texttt{in1} of our entity.
    Now we need to simulate our code again, by analyzing, elaborating and running it again to generate a \texttt{.vcd} file.
    \begin{lstlisting}
    1. ghdl -a and_gate.vhd
    2. ghdl -a process.vhd
    3. ghdl -e -fsynopsys testbench
    4. ghdl -r testbench --vcd=output.vcd --stop-time=25s
    5. vcd < output.vcd
    \end{lstlisting}
\end{aufgabe}

\begin{aufgabe}
    In the final example of this worksheet, clone the following code, and explain what it does: \url{git@gist.github.com:56a39b44311535a8b2fec3b29763e57d.git}.
    Try running the simulation again.
\end{aufgabe}

\section*{Annotations}
\begin{enumerate}
    \item VCD Command Line Viewer: \url{https://github.com/yne/vcd}
\end{enumerate}

\end{document}
